\documentclass{beamer}
\usepackage[no-math]{fontspec}
\usepackage{xeCJK}
\setCJKmainfont{Source Han Sans TW}

\usetheme{CambridgeUS}
\title{Turner syndrome}
\author[Chen-Pang He]{何震邦 (Chen-Pang He), Intern}
\institute[CGH]{Cathay General Hospital}
\date{December 24, 2018}

\begin{document}
\maketitle

\begin{frame}{Cause}
    \begin{itemize}
        \item Complete or partial monosomy of the X chromosome
        \item Half of the patients are 45,X
        \item 15\% are 45,X/46,XX
        \item Other mosaics are less often
            \begin{itemize}
                \item Isochromosomes, rings, fragments
            \end{itemize}
        \item In 75\% of patients, the lost sex chromosome is paternal.
            \begin{itemize}
                \item Maternal age is not a predisposing factor
            \end{itemize}
        \item 1/2000--1/5000 females at birth
    \end{itemize}
\end{frame}

\begin{frame}{Signs and symptoms}
    \begin{itemize}
        \item Recognizable at birth
        \item Short stature
        \item Cardiac defects
        \item Renal malformations
        \item Autoimmune thyroid diseases
        \item Inflammatory bowel diseases
        \item Sternal malformation
        \item Recurrent bilateral otitis media
    \end{itemize}
\end{frame}

\begin{frame}{Diagnosis}
    \begin{itemize}
        \item Prenatal
            \begin{itemize}
                \item Amniocentesis or chorionic villus sampling
                \item Abnormal ultrasound findings
            \end{itemize}
        \item Postnatal
            \begin{itemize}
                \item Karyotyping
                \item Webbed neck and lymphedema at birth
            \end{itemize}
    \end{itemize}
\end{frame}

\begin{frame}{Treatment}
    \begin{itemize}
        \item Recombinant human growth hormone
            \begin{itemize}
                \item Boost adult height
            \end{itemize}
        \item Estrogen replacement therapy
            \begin{itemize}
                \item Age-appropriate puberty
                \item Preventing cardiovascular events
                \item Preventing osteoporosis and scoliosis
                \item Occasionally restoring reproductive function
            \end{itemize}
    \end{itemize}
\end{frame}

\end{document}
